% Per generare il pdf della tesi compilare questo file .tex, poi aprire Frontispiece-frn.tex, 
% compilare anche quello e infine ritornare qui per compilare una seconda volta questo documento.
% Una volta generato, il pdf del frontespizio sarà incluso nel file principale con un comando
% \includepdf

\documentclass[a4paper, titlepage]{book}
\usepackage[utf8]{inputenc} % UTF8 per avere tutti gli accentacci
\usepackage{frontespizio} 	% http://texdoc.net/texmf-dist/doc/latex/frontespizio/frontespizio.pdf

\begin{document}
	\begin{frontespizio}
		\Istituzione{Università di Pisa}
		\Divisione{Dipartimento di Ingegneria dell'Informazione}
		\Scuola{Laurea Triennale in Ingegneria Informatica}
		\Logo[3.5cm]{cherubino_pant541_svg.pdf}
		\Titolo{Progettazione ed implementazione di un sistema di raccolta e analisi di informazioni di cybersecurity}
		%\NCandidato{Candidate}			% Togliere il commento per frontespizi in inglese
		\Candidato{Dario Capecchi}
		%\NRelatore{Advisor}{Advisors}	% Togliere il commento per frontespizi in inglese
		\Relatore{Prof. Mario G. C. A. Cimino}
		\Relatore{Prof. ssa Gigliola Vaglini}
		\Relatore{Dott. Giacomo Giorgi}
		\Piede{Anno Accademico 2020-2021}
		\Margini{3cm}{3cm}{3cm}{3cm}
	\end{frontespizio}
\end{document}
